
%
% LaTeX2e template for FIT2002
%


\documentclass[a4j,twocolumn]{jarticle}
\usepackage[dvips]{graphicx}
\usepackage{here}
\usepackage{url}


\makeatletter
\def\section{\@startsection{section}{1}{\z@}{2ex plus .2ex minus .2ex}%
      {.5ex plus .2ex minus .2ex}{\large\bfseries}}
\def\thesection{\arabic{section}.}
\def\subsection{\@startsection{subsection}{1}{\z@}{.7ex plus .2ex minus .2ex}%
      {.5ex plus .2ex minus .2ex}{\normalsize\bfseries}}
\def\thesubsection{\arabic{section}.\arabic{subsection}}
\def\thefootnote{\fnsymbol{footnote}}
\makeatother

\def\baselinestretch{0.8}

\setlength{\textheight}{26.5cm}%297-30-27 - 5
\setlength{\textwidth}{17.4cm}%210-18-18 - 10
%\setlength{\headheight}{0.0in}
\setlength{\headsep}{0.0in}
\setlength{\oddsidemargin}{-.9cm}%+3
\setlength{\evensidemargin}{-.9cm}%+3
\setlength{\columnsep}{7mm}
\usepackage[dvipdfmx]{graphicx}
% local settings

% end of local settings

\begin{document}
\pagestyle{empty}
\thispagestyle{empty}

\twocolumn[%
\begin{center}
 {\Large MyRecipe:文字書き換え機能を使用した\\料理初学者向けレシピ利用補助システムの開発}\vspace{.5ex}
 {\Large\sffamily }\vspace{1ex}

\large
\mbox{}
\hfil
\setcounter{footnote}{2}
{\bfseries Life-Cloud B3 岩淵 駿(iwatchi)}${}^\thefootnote$
\hfil
\setcounter{footnote}{3}
{\bfseries Adviser:江頭 和輝(zukky)}${}^\thefootnote$
\hfil
\mbox{}

\mbox{}
\hfil
{\sffamily}
\hfil
{\sffamily}
\hfil
\mbox{}
\hfil

\end{center}
]

\setcounter{footnote}{2}
\footnotetext{慶應義塾大学 環境情報学部}
\setcounter{footnote}{3}
\footnotetext{慶應義塾大学 政策/メディア研究科}

\section{概要}
  クックパッドのレシピページの材料を変更し,料理初学者が調理しやすいレシピの内容を作成するレシピ改良システムを提案する.
\section{背景}
クックパッドの2016年の第二四半期で発表された集計結果\cite{cpd2016}によると,PC,スマホ,ガラケーの利用者数の合計は,約6,109万人となっている.また,2013年7月のインターネットコムの調査によると,料理をする際の「もっとも参考にする情報源」は,レシピサイト(57\%)が,料理の本(15\%)やテレビ(9\%)を突き放して圧倒的トップとなっていた。このことから,レシピサイト/アプリは広く普及したと言える.\cite{sharemaru2015}
\section{問題意識}
  クックパッド上では多数のレシピを参照できるが,一部のレシピに使われる材料は知名度や普及率が低いため,料理初学者の既知の食材にない場合がある.このような食材は一般的なスーパーに置いていないことも多い.そのため,実際に料理を行う際の妨げとなる場合がある.また,多くのレシピに見られる抽象的な表現(例:きつね色の焼きかげん,耳たぶの硬さの生地)も料理初学者には基準を把握することが難しい.これらは調理の敷居を高くし料理初学者が調理を挫折する要因となる.そこで本研究の目的は文字書き換え機能を使用することで,レシピの特定の食材名を書き換え,抽象的な表現を修正することでレシピの実現のしきい値を低くすることである.

\section{MyRecipeの開発}
  MyRecipeは,クックパッド上のレシピページの単語を置き換えることで,レシピの文中に登場する抽象的な表現や,知名度の低い食材といった料理初学者の障害になる要素を変更し,レシピの利用のハードルを下げ,実現可能性を広げる利用補助システムである.
\subsection{予想使用例}
  レシピ中には「きつね色の焼き加減」や,「耳たぶのかたさまで生地をこねる」といった表現が使われる.これらを「明るいオレンジ色」,「弾力性がありちぎりにくい」などの表現に置き換えることで,調理の敷居を低くする.同時に,「タロイモ」などの知名度の低い食材も料理のレベルをあげているため,これらを「里芋」など知名度の高い材料に置き換えることでレシピを利用しやすくすることでレシピの実現難易度を低くすることがねらいである.

\subsection{アプローチ}
  アプローチとして,単語書き換えモジュールを使用しレシピページの単語を書き換える.
\par 手順として,最初にクックパッドのレシピページを読み込み,ページ解析モジュールによって単語を抽出する.抽出した単語を単語書き換えモジュールに通し,テキスト情報に変換して単語のテキスト情報を格納する辞書と照合を行う.抽出された単語と同じ合致する単語が辞書に見つかった場合,単語を書き換える.辞書には,テキストで作成し,書き換え前の単語は「before」,書き換え後の単語は「after」として格納しておく.書き換え後,for文でウェブページ上の抽出されたすべての単語の照合と書き換えが終了するまで繰り返し,処理を行う.書き換えの処理が終了したら表示モジュールによって内容を表示し,ユーザにフィードバックする.
\begin{figure}[H]
\centering
\includegraphics[scale=0.3]{si2.eps}
\caption{MyRecipe使用実例(「生クリーム」書き換え後)}
\end{figure}


\section{実装環境}
  アプリケーションの実装にはjavascriptを使用する.また,ブラウザにはGoogle Chromeを使用する.システム構成図は図1に示す.
\begin{figure}[htbp]
\centering
\includegraphics[width=\linewidth]{sys1.eps}
\caption{システム構成図}
\end{figure}

\section{リミテーション}
 今回の研究ではデータベースなどを用いず手動で作成した辞書を使用しているため,置き換えられる単語の数に限りがある.次に,特定の食材がレシピにおいて主食であるか香り付けに使用されるかなどの個別のケースには対応していない.最後に,レシピ利用者の習熟度に合わせ特定のレシピのみ内容を変更することはしていない.将来の研究でこの3つに取り組みたいと考えている.

\section{まとめ}
  文字書き換えによってクックパッド上のレシピの利便性を向上する料理初学者向けインターネットレシピ利用補助システムを提案した.

% \bibliographystyle{junsrt}
% \bibliography{test.bib}

\begin{thebibliography}{5}
  \bibitem {cpd2016} クックパッド決算説明資料2016第二四半期
  \url{http://cf.cpcdn.com/info/assets.html}
  \year 最終閲覧日 2017.01.17
  \bibitem {cpd2014} クックパッド決算説明資料2014第二四半期
  \url{http://cf.cpcdn.com/info/assets.html}
  \year 最終閲覧日 2017.01.17
  \bibitem {sharemaru2015} クックパッドの利用率光る/レシピシェアサイト早わかり
 \url{http://ascii.jp/elem/000/000/951/951053/index-2.html}
 \year 最終閲覧日 2017.01.17
\end{thebibliography}


\end{document}
